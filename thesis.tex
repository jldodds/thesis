\documentclass{puthesis}
\usepackage{latexsym}
\usepackage{graphicx}
\usepackage{url}       % SB
\usepackage{algorithmic}
\usepackage{algorithm}
\usepackage{times}
\usepackage{xcolor}
\usepackage{textcomp}
\usepackage{mathpartir}
\usepackage{semantic}
\usepackage{listings}
\usepackage{lstlangcoq}
\usepackage{caption}
\usepackage{stmaryrd}
\renewcommand{\lstlistingname}{Figure}
\newcommand{\PROP}{\mbox{\small PROP}}
\newcommand{\LOCAL}{\mbox{\small LOCAL}}
\newcommand{\SEP}{\mbox{\small SEP}}
\newcommand{\later}{\triangleright}
\newcommand{\wand}{\mathrel{-\hspace{-.7ex}*}}
\newcommand{\triple}[3]{\{#1\}\,#2\,\{#3\}}
\newcommand{\name}{Verified Software Toolchain}
\newcommand{\defeq}{=_{\mathrm{def}}}


\lstset{language=Coq,basicstyle=\sffamily,mathescape=true,columns=fullflexible}



\author{Josiah Dodds}
\adviser{Andrew Appel}
\title{Computation Improves Interactive Symbolic Execution}
\abstract{The abstract goes here.}
\acknowledgements{Thank you very much.}
\dedication{To myself.}



\begin{document}



\chapter{Introduction}

\chapter{Computation in Coq}

Coq's functional programming language Gallina has the ability to match inductive
data-structures. The type boolean, for example can be defined as

\begin{lstlisting}
Inductive boolean :=
| true
| false.
\end{lstlisting}

so \lstinline|boolean| is a type that can be constructed by either \lstinline|true| or \lstinline|false|
and when given a \lstinline|boolean|, a function can decide which of the two it is. Now we can look 
at the definition of \lstinline|True| and \lstinline|False| (capitalized), which are in the type 
\lstinline|Prop| instead of \lstinline|bool|. Prop is not an
inductively defined type in Coq, so it can't be matched
against. Instead the types of \lstinline|True| and \lstinline|False|
are

\begin{lstlisting}
Inductive True : Prop :=
  I : True.

Inductive False : Prop
\end{lstlisting}

So if a Coq program is given something of type \lstinline|Prop|, it
has no ability to determine what it is, just like if it is given a
value of an arbitrary type it is unable to determine what that type
is. That doesn't mean that it is impossible to reason about things of
type \lstinline|Prop| in Coq though. Instead, Coq has tactics, which
exist almost exclusively for that reason. A tactic in Coq is a program
that manipulates a proof state. A Coq proof state is a goal, or a
\lstinline|Prop| to be proved, along with a number of quantified
variables of any type, and a number of hypothesis of type
\lstinline|Prop|. Because Ltac is used to build proofs, it must keep a
record any time it makes progress on the proof term so that when the
proof is over, the record of the proof, or the proof object, can be
checked.

This design means that Ltac doesn't need to be sound. As long as it
results in a correct proof in the end, it doesn't really matter how it
did it. Proofs that do a large amount of proof search to do a
reasonably small amount of work in the end can perform reasonably well
in Ltac. Unfortunately, many of the proofs that Coq's users want to do
require both a lot of proof search, and numerous operations over the
proof state to succeed. In these cases, the overhead of keeping and
modifying the
proof object (which can take up substantial amounts of memory) can
lead to very slow Ltac performance. 

Logics such as VST's verifiable C logic are in
\lstinline|Prop|. Logics in Prop are said to be shallowly
embedded. They use the syntax of Coq, rather than defining a logic of
their own. Some systems such as Appel's VeriSmall \cite{} are deeply
embedded, meaning they define their own syntax, along with a
denotation that gives meaning to that syntax in Coq's logic. 

Shallow and Deep embeddings have tradeoffs in two areas:

\begin{enumerate}
\item interactivity, or how convenient it is for users to interact
  with the logic, generally in the way they would expect to be able to
  interact with Coq's logic, and
\item automation, or the ability of the logic writer to provide the
  user of the logic with tools to efficiently reason about the logic.
\end{enumerate}

For interactivity, a shallow embedding will automatically get all of
the automation that Coq provides. Ltac can be used to write decision
procedures about the logic without requiring soundness proofs for
those procedures. A deep embedding, however, will be much harder to
interact with. To have meaning as a proof goal, a deep embedding will
need to be wrapped in its denotation function. At this point many of
the operations users familiar with Coq would expect will require a
significant amount of work from the creator of the logic. The main
reason for this is that in a deep embedding every operation on the
deeply embedded assertion must be proved sound with respect to the
denotation function. Take Coq's rewrite tactic as an example. In this
tactic if we have a lemma about an equality for example:

\begin{lstlisting}
Lemma add_symm : forall a b, a + b = b + a
\end{lstlisting}

and we wish to prove a goal 

\begin{lstlisting}
a : nat
b : nat
c : nat
======================
a + (b + c) = a + (c + b) 
\end{lstlisting}

we can use the tactic \lstinline|rewrite (add_sym b c); reflexivity|
to transform the left side to match the right side, and then tell Coq
that we have an equation where the two sides are syntactically
equal. We supply the arguments \lstinline|b| and \lstinline|c| to the
tactic because it is ambiguous where to rewrite the lemma, and Coq
might guess wrong unless we tell it.

Now imagine we have a deep embedding for nat, addition, and equality

\begin{lstlisting}
Inductive expr' :=
| num : nat -> expr
| add : expr -> expr -> expr.

Inductive expr :=
| eq : expr' -> expr' -> expr.

Fixpoint expr'_denote e:=
match e with
| num n => n
| add e1 e2 => expr_denote e1 + expr_denote e2
end.

Definition expr_denote e :=
match e with
| eq e1 e2 => expr'_denote e1 = expr'_denote e2
end.
\end{lstlisting} 

Then we can write a symmetry lemma

\begin{lstlisting}
Lemma add_sym' : forall a b, 
expr'_denote (add a b) = expr'_denote (add b a)
\end{lstlisting}

While this isn't too hard to prove, it also doesn't end up being too
useful in proving our goal:

\begin{lstlisting}
a : expr
b : expr
c : expr
======================
expr_denote (add a (add b c)) (add a (add c b))  
\end{lstlisting}

The lemma we wrote doesn't match the syntax of what we need to prove
so we can't use the rewrite tactic. We could unfold the definition of
\lstinline|expr_denote| and \lstinline|expr_denote'| in our goal, but
then we would lose the deep embedding. Instead, if we want this
functionality, we need to write a \emph{Coq function} that does the
rewrite and prove that function sound with respect to the denotation
function (assuming we have an equality function on our expr, which
isn't hard to write):

\begin{lstlisting}
Fixpoint symmetry' e e1 e2 := 
match e with
| add e1' e2' => if (e1 == e1' && e2 = e2') 
                 then add e2' e1' 
                 else add (symmetry e1') 
                          (symmetry e2')
| x => x
end.

Definition symmetry e e1 e2 :=
match e with
| eq ex1 ex2 => eq (symmetry' ex1 e1 e2) (symmetry ex2 e1 e2)
end.

Lemma symmetry_func_sound : 
forall e1 e2,
expr_denote (e) = expr_denote (symmetry e)
...
\end{lstlisting}

Now we can rewrite by \lstinline|symmetry_func_sound| and simplify the definition
of symmetry in the new goal, and we will have done the same thing we
did with LTac in a single command.

Why would we ever use a deep embedding when automation is so much
work? The main reason is efficiency. While LTac does an operation and
needs to build a proof object, a proved-sound function that operates on
a deeply embedded proof goal does an operation and at the same time is
a (very small) proof object. The example above did a very small amount
of work in the symmetry function, but such a function could be an
entire decision procedure. Then we would be able to do a large
amount of work on a deeply embedded proof term while hardly generating
any proof object at all. In general, this will be substantially more
efficient than using LTac, especially as the size of the
deeply-embedded statements grows. 

Another advantage to a deeply embedded logic is that if there is a
single decision procedure for it that works all in one function, that
function can be \emph{extracted} and run in OCaml. This gives the
performance of an optimized language, with most of the assurance that
comes with having a proved-sound decision procedure in Coq.

The technique of computational reflection in Coq aims to combine the
interactivity of a shallow embedding with the efficiency of a deep
embedding by moving between the two when it is needed.

\section{Computational reflection}

Computational reflection is the process of translating a shallow
embedding to a deep embedding, applying a proved-sound function to the
deep embedding, then evaluating the denotation function to return to
the shallow embedding. If this is done all in a single step, the user
will never know that the technique of reflection is being used. That
means that the efficient, proved-sound decision procedures that can be
used on deep embeddings can essentially be used as if they are
tactics, along side all of the other tactics and LTac automation that
might be built up around a logic. 

The process of translating from a shallow embedding to a deep
embedding is known as reification. Because Coq programs can't match on
shallow embeddings, reification must be performed by a
tactic. Reification leaves us with a deep embedding (or reified
expression), which we can do computation on in the same way

\chapter{Typechecking C Expressions}
think about if there is more that needs to be explained

\chapter{Canonical Forms for Assertions}

VST uses assertions to reason about programs, but what does 
it require about the form of these assertions? This chapter
discusses that, as well as how restricting the form of
assertions improve both the usability and the performance
of symbolic execution.

\section{Semi-Canonical Form}

We previously described \cite{appel14:plcc} assertions in the form
~~\lstinline{PROP  $~P$ LOCAL  $~Q$ SEP  $~R$}. 
Here each item in \lstinline|$P$:list prop| is a pure assertion that 
doesn't refer to the program state. \lstinline|$Q$: list (environ -> prop)|
contains assertions that can reason about local variables. 
and \lstinline|$R$: list (environ -> mpred)| has spatial assertions that
can reason about both local variables and memory. 
$\PROP$ folds conjunction over the elements of $P$, $\LOCAL$
folds lifted conjunction (
\lstinline|`and : (environ -> Prop)-> (environ -> Prop) -> (environ -> Prop))|,
and $\SEP$ folds the separation logic \lstinline|*|, or separating conjunction
operator. $P$, $Q$, and $R$, are represented as lists because, particularly 
in the case of $\SEP$, it is often convenient to refer to the $n$th 
conjunct. This is much easier to implement when our assertion is a list. 
We say that assertions in this form are in semi-canonical form. 

At the lowest level, the Verifiable C 
logic rules are completely unaware of any
sort of canonical form. They generally refer to assertions as single
variables, using entailments to constrain them rather than imposing
syntactic requirements on them. The Floyd automation system contains
higher level lemmas that require assertions to be in semi-canonical
form, but also guarantee that the side conditions that result from
using the rules will be in semi-canonical form. 

\subsection{Substitution in semi-canonical form}

At the Verifiable C level, there is no choice but to use a semantic
notion of substitution called 
\lstinline|subst {A: Type} (x : ident) (v:val) (P : environ -> A) : environ -> A|.
This is because we know nothing about the assertion at all, only that
it takes an environment and returns an \lstinline|mpred|. The following example
shows how \lstinline|subst| is used:

\begin{lstlisting}
$\inference[semax\_set\_forward]{}{
\Delta\vdash\triple{\later P}{~x:=e~}{\exists v.\,x=(e[v/x])\wedge P[v/x]}
}$

Axiom semax_set_forward: $~~$forall $\Delta$ ($P$: environ->mpred) ($x$: ident) ($e$: expr),
  semax $\Delta$
    (|> (local (tc_expr $\Delta$ $e$) && local (tc_temp_id id (typeof $e$) $\Delta$ $e$) && $P$))
    (Sset $x$ $e$) 
    (normal_ret_assert 
      (EX old:val, local (`eq (eval_id $x$) (subst $x$ (`old) (eval_expr $e$)))
                    && subst $x$ (`old) $P$)).
\end{lstlisting}

There are two substitutions here, used to replace any occurrences of the
variable \lstinline|x| that might have occurred in either the precondition
or the expression being assigned into \lstinline|x|. 
Although subst is a function, in practice it can never be computed.
This is because it works by updating the environment that $P$
refers to. During symbolic execution, however, the environment is always
abstract, constrained only by the precondition, which means there is
no datastructure to be updated. This means that the definition of 
\lstinline|subst| that appears in Verifiable C isn't directly 
useful to proof automation. It can't compute so without special
lemmas and tactics it will appear in side conditions. To deal with
this the Floyd system has an autorewrite database that lets it push
subst through functions that won't be affected by the substitution. For
example 

\begin{lstlisting}
Lemma subst_sepcon: forall i v (P Q: environ->mpred),
  subst i v (P * Q) = (subst i v P * subst i v Q).
\end{lstlisting}

Fortunately, we don't need a lemma for every function that might
appear in assertions. Lifted functions can't do anything with the
environment, they can only pass it on to their arguments, so
by creating autorewrite rules for lifted functions we cover
most of the functions that we use, and also most functions
that a user might want to write. 

Semantic substitution is still inconvenient for a few reasons. First,
the rewrite rules aren't complete. This means that in some cases, after
applying a logic rule, the user will see a \lstinline|subst| in a
resulting condition. This can stop the automated entailment
solvers from working correctly and make the assertion much harder
to read. The next problem is an issue with autorewrite in general.
Autorewrite in Coq is slow. Rewrites aren't known for their 
performance, and autorewrite can do a large number of rewrites
(in the case of \lstinline|subst| the number of rewrites is
linear in the size of the assertion being rewritten). 

There is a situation when a substitution \lstinline|subst $x$ $v$ $P$| can
be avoided completely. That is when $P$ is \emph{closed} wrt. 
$x$, also a semantic notion:

\begin{lstlisting}
Definition closed_wrt_vars {B} (S: ident -> Prop) (F: environ -> B) : Prop := 
  forall rho te',  
     (forall i, S i \/ Map.get (te_of rho) i = Map.get te' i) ->
     F rho = F (mkEnviron (ge_of rho) (ve_of rho) te').
\end{lstlisting}

Generally we give \lstinline|S| as Coq equality with a specific identifier. 
What \lstinline|closed_wrt_vars| means, then, is that if \lstinline|F|
is supplied an environment that is the same at all locations but 
the identifier(s) \lstinline|i| that satisfy \lstinline|S|, the
result of \lstinline|F| will be the same. That means that if we know
\lstinline|closed_wrt_vars (eq x) (e)|, we can easily prove
\lstinline|subst x _ e = e|. More intuitively, if an expression
doesn't contain a variable, a substitution on that variable
won't change the expression. 

Floyd has a set lemma that takes advantage of this, stating that if
the precondition and the expression in the assignment
are closed wrt the variable being assigned into, no substitutions are
needed, but there are numerous cases where this rule doesn't apply, 
so the substitution will still appear. 

\section{Canonical Form}

The reason that substitutions are difficult, and that they need
to be semantic is because there is no \emph{syntactic} restriction
on where any individual identifier can appear within an assertion.
Canonical form imposes such a restriction, and in doing so, eliminates
the need for semantic substitution, replacing it with a more
efficient and convenient computational syntactic substitution.

One limitation of canonical form is that we no longer allow
references to C program variables in the part of the assertion that
contains spatial assertions. If these assertions wish to talk
about those variables, they must do it indirectly using a Coq variable.
This means that only the $\LOCAL$ part of the assertion has the ability
to reference local variables. This still doesn't give us the ability to
syntactically locate each reference though, so we restrict $\LOCAL$
further. The restriction we use is to change the entirety of 
$\LOCAL$ into two computational maps from identifiers to values.
One of these represents temporary or nonadressable variables, and 
the other represents addressable variables. Each mapping represents
an equality between the evaluation of an identifier in the
environment, and the value it maps to. The mappings are represented by PTrees,
an efficient
computational data structure in the Coq standard library. 

With these two changes we get \emph{canonical form.}
Let \lstinline{$T_1$: PTree val} be a computational map from
C program identifiers to C values,
representing the current values of the temporary local
variables of the current program state. Let
\lstinline{$T_2$: PTree (type*val)}
be a map from identifiers to \lstinline{type*val}
representing the addresses of addressable local variables.
Then \lstinline{localD $T_1$ $T_2$: list(environ->Prop)}
means a list of assertions about the contents of the \lstinline{environ},
the nonmemory portion of the program state;
we do not need \emph{arbitrary} 
assertions of type \lstinline{list(environ->Prop)}.

\lstinline{localD} is a \emph{denotation function},
reflecting the syntactic (computationally oriented) $T_1$ and $T_2$
back into our semantic world.  In symbolic execution
and efficient entailment solving, we operate directly on 
$T_1$ and $T_2$, reflecting the results back only when the 
less efficient (but easier to understand) semantic view is
needed by the user. Now a full assertion is:

\begin{lstlisting}
assertD $P$ (localD $T_1$ $T_2$) $R$ : environ->mpred
$P$ : list prop$\qquad$ $T_1$ : PTree val$\qquad$ $T_2$ : PTree (type * val)$\qquad$ $R$ : list mpred
\end{lstlisting}

\subsection{Substitution in Canonical Form}

Substitution in this assertion is 
as simple as adding/replacing a mapping in $T_1$ or $T_2$. To see why imagine
that we are doing a substitution on a temporary variable $x$. 
\lstinline|$P$ : list prop| and \lstinline|$R$ : list mpred| don't
refer to an environment, so they are trivially closed wrt.
$x$. This leaves $T_1$ and $T_2$. The variable $x$
is a temporary variable, so we know $T_2$ is closed wrt. $x$.
The map $T_1$, however, might have a reference to $x$, meaning
we actually need to do a substitution, making sure to replace every reference
to that variable. One of the requirements of a Coq map is:

\begin{lstlisting}
Axiom PTree.gss
     : forall (A : Type) (i : positive) (x : A) (m : PTree.t A),
       PTree.get (PTree.set i x m) i = Some x
\end{lstlisting}

This means that if we update $x$ in some PTree, the old
mapping of $x$ will no longer exist, which is the exact
definition we want from a substitution. That is how you do a substitution
in an assertion, but we still need to do substitution in the arbitrary
C expression that appears in the assignment statement and turn that expression
into a value. It is simple enough to turn a C expression
into an \lstinline|environ->val| using the \lstinline|eval_expr| function
discussed in \ref{}, but that expression could have references to identifiers,
which we can't have if we want syntactic substitution. Instead we can write
a different version of \lstinline|eval_expr| called 
\lstinline|msubst_eval_expr|. The only difference between the two functions
is that when \lstinline|msubst_eval_expr| needs to evaluate a variable it
doesn't do it in and environment. Instead, it performs the lookup in 
PTrees $T_1$ or $T_2$ mentioned earlier. In other words, if
\lstinline|eval_expr| evaluates an expression in an environment, 
\lstinline|msubst_eval_expr| symbolically evaluates an expression
in an assertion. This symbolic evaluation is partial because there might
not be any information about a variable in the assertion. So in our
lemma we require \lstinline|msubst_eval_expr| to succeed:

\begin{lstlisting}
Axiom semax_PTree_set: $~~$forall $\Delta$ id P T1 T2 R $e$ v,
  msubst_eval_expr T1 T2 $e$ = Some v ->
  semax $\Delta$
    (|> local (tc_expr $\Delta$ $e$) && local (tc_temp_id id (typeof e) $\Delta$ e) 
            && (assertD P (localD T1 T2) R))
    (Sset id $e$)
    (normal_ret_assert (assertD P (localD (PTree.set id v T1) T2) R)).
\end{lstlisting}

What that means is that for this lemma to be used, the precondition must
have mappings for every variable that appears in $e$. The previous 
lemma didn't require this because it was able to use \lstinline|eval_expr|
wherever it wanted to without requiring a complete, successful, symbolic
execution. This still works for proofs because \lstinline|`eval_expr $e$|
might eventually simplify to \lstinline|`eval_id $x$|, which could appear
in other places in the assertion. 

\lstinline|semax_PTree_set| also doesn't have an existential. This
makes things simpler for the user and the proof automation. The existential
for the old value isn't terribly inconvenient on it's own, it can be 
moved to the outside of the triple and introduced without much difficulty,
especially because it's location in the precondition is consistent. The
difficulty comes when choosing what to name the introduced variable. The
solution in the tactics is to allow the user to specify names with the name
tactic. Using \lstinline|name y _y|) tells the automation that values
associated with variable \lstinline|_y| should automatically be named
\lstinline|y| or \lstinline|y0|, \lstinline|y1|, \ldots if it isn't available.
This is a decent solution but it puts a hypothesis above the line, adding
to what can already be a long list of hypotheses. It is also inconvenient
in programs that make multiple assignments into the same variable. The
following program is an example of what you might see without improved
tactics or user cleanup: 

\begin{lstlisting}
{`eq (eval_id _x) x}
x = x+1;
{`eq (eval_id _x) (x0 + 1); `eq (x0 x)}
x = x+2;
{`eq (eval_id _x) (x1 + 1); `eq x1 (x0 + 1); `eq x0 x}
...
\end{lstlisting}



Semi-canonical form is very convenient for the user when \emph{writing}
assertions. The list notation is great for combining 
assertions without having to remember the exact conjunction that
must be used for each part. It also allows the $\LOCAL$ to remain
small, because if there is a variable the user knows nothing
about, there is no need to add it to the locals. It is less convenient
when moving through a proof of a program. It can introduce
existentials and substitutions that are slow to simplify, or sometimes
don't simplify at all. 

Even the current form is not completely canonical. It could be
restricted further which would improve performance in some ways, but
also inconvenience the user in others. Finding a way to sort the
$\SEP$ and keep it sorted as new conjunctions are added could lead to
very efficient and simple entailment solving. This is a harder problem
than sorting the $\LOCAL$ though, because we want $\SEP$ to contain a
variety of predicates, including predicates that are created by the
user. The other problem is that Coq variables can appear as arguments
to the predicates, making it impossible to establish an ordering over
them. Even so, any amount of canonicalization of the $\SEP$ could help
with entailment solving efficiency, so it is worth a look in the future. 

\chapter{Applying A Reflective Framework}

\section{Modular Reflection}

Before presenting the reflective framework we presented both the typechecker,
and a canonical form for assertions which allows us to do substitutions computationally.
Both our techniques and the reflective framework are reflective.
They use proved-sound Coq functions to make progress in a proof. This
section discusses how these techniques interact. There
is no interesting interaction between the substitution and the
typechecker because they occur in different places. When we try
to apply a reflective framework to a logic that uses these reflective
techniques, though, we run into some interesting challenges. This
section discusses those challenges and questions if they are 
avoidable in a differently designed reflective framework.

A function that is used in a logic can fall into one of
three categories with respect to their interaction
with the reflective framework:

\begin{itemize}
  \item A function that operates on constants and returns a type that can be represented as a constant
    requires no modification to be used in MirrorCore.
  \item A function that operates on constants and returns Prop or Type will look exactly
    the same but return reified results.
  \item A function that might operate on Coq variables must take reified expressions as 
    arguments and return a reified result.
\end{itemize}

The first category is convenient, but unfortunately fairly few functions that
we use fall into this category. The reason that they can return any
type but Type or Prop is because in general if a function returns a result we
will want to reason about it, and if it can't be represented by a constant
we will be unable to do anything with it in the Mirror-Core framework.

The second category includes the typechecker which returns
\lstinline|environ -> Prop| that might eventually be discharged
by either the person writing the proof or some automation. 
In order to allow reflective automation to have a chance to solve these remaining
conditions, we must write a new function who instead of returning
\lstinline|environ -> Prop| returns an \lstinline|expr| whose denotation
is (provably) the same as the result of the original typechecker called
on the same arguments.

\begin{lstlisting}
Definition tc_expr_reif (e : c_expr) (Delta : tycontext) : expr.

Lemma tc_expr_reif_sound_complete : forall tus tvs e Delta,
exprD' tus tvs (tyarr tyenviron typrop) (tc_expr_reif e Delta) =
exprD' tus tbs (tyarr tyenviron typrop) (ftc_expr e Delta).
\end{lstlisting}

Notice that the right hand side of the equality in the lemma doesn't
refer to the original typechecking function but the trivially reified
version. That is, \lstinline|ftc_expr| is a constructor whose denotation is equal
to \lstinline|tc_expr|, while \lstinline|tc_expr_reif| is a function that when applied will have
a denotation equal to \lstinline|tc_expr| applied to the same arguments. The function
\lstinline|tc_expr_reif| can compute while no progress can be made on \lstinline|ftc_expr|.
We put the denotation function on both sides of the function because
this allows us to write an RTac that finds reified terms that look like
\lstinline|ftc_expr e Delta| (uncomputable) and replace them with a term
\lstinline|tc_expr e Delta|. This is important because we might want
to run some other RTac on the result of the typechecking. Possibly
an entailment solver to discharge any remaining conditions. 

We prove that \lstinline|tc_expr_reif| is sound, or that when it has a denotation,
that denotation matches the \lstinline|tc_expr| function. We also prove it complete by
showing that whenever \lstinline|ftc_expr| has a denotation, \lstinline|tc_expr_reif| does as well.
In this case sound and complete is the most useful so we prove it. We have
found other cases where only soundness is necessary, and will discuss those
later.

The final category is functions whose inputs might contain Coq variables. An example
of where we run into this is the \lstinline|PTree.set| and \lstinline|PTree.get|
operations that are in the postcondition of our assignment rules. These functions
can be computed on their own, even if they are given a \lstinline|PTree| with a
coq variable in it as an argument. The problem is that there is no way to represent
any datastructure that contains a coq variable as a constant. This is because there
is no way to computationally compare two Coq variables for equality. In order to 
computationally check equality over Coq variables, they must first be reified.
Reification is designed so that if it sees the same variable twice, it will use
the same constructor to represent both instances. The constructors can be 
computationally compared, so variables can be compared in reified syntax.
We will need to do comparisons on the variables in the local assertion almost
any time we solve an entailment, so we will need to reify the variables.

What that means is that we must rewrite \lstinline|PTree.set| and \lstinline|PTree.get|
to operate on fully reified PTrees. This can be difficult because of the size 
of the reified syntax. One way to mitigate the complexity of the code is to 
create a function that matches a reified expression as a tree:

\begin{lstlisting}
Definition as_tree (e : expr typ func) : option
  ((typ * expr typ func * expr typ func * expr typ func) + typ) := 
match e with
  | (App (App (App (Inj (inr (Data (fnode t)))) l) o) r) =>
    Some (inl (t, l, o, r))
  | (Inj (inr (Data (fleaf t)))) =>
    Some (inr t)
  | _ => None
end.
\end{lstlisting}

This allows the function that operates over reified expressions to look very
similar to original expression, and also simplifies the proof. The function
and the proof will now have 3 cases. The first have the same behavior (only reified)
as the \lstinline|leaf| and \lstinline|node| cases of the original function. The third
case is the case where \lstinline|as_tree| returns \lstinline|None|. This doesn't
mean that the function doesn't typecheck because there is no requirement on 
expressions that a PTree can only be represented by those two constructors. There
could be another constructor with the same denotation, or more likely a PTree
could be an application of \lstinline|PTree.set| that didn't get simplified.
If we encounter a reified PTree that isn't represented
only by the expected constructors, we have a choice depending on
weather we require completeness or not. If we require completeness,
we can just apply the set function as a reified constructor. This
will result in a complete function. That means that if we
pass our reified function all valid reified arguments
the denotation of the result of the function will always
match the denotation of the original function applied
to the denotation of the arguments. It is the most convenient
to write an RTac for reified functions of this type because
there is a pre-built RTac that replaces one reified expression
with another. This RTac can only be proved sound if there is a proof
of equality between any two expressions that the tactic might come
across. The only downside to this approach is that proving that the
reified function is complete takes much more work than only proving it
sound. 

These reified functions are an inconvenience and one of the largest
barriers to easily applying a reflective framework in a setting like
VST. It is an important open question weather this is avoidable. Can
a reflective framework reuse computations that might operate
over Coq variables without requiring reified functions? 

\chapter{Evaluation}

\chapter{Conclusion}

\bibliographystyle{plain}
\bibliography{appel.bib}

\end{document}

